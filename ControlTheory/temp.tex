\documentclass[a4j]{ujarticle}
\usepackage[dvipdfmx]{graphicx}
\usepackage{url}
\usepackage{bbding}
\usepackage{lscape}
\usepackage[subrefformat=parens]{subcaption}
\usepackage{bm}
\usepackage{amsmath}

\title{進捗報告資料}
\author{安達智哉\\to-adachi@ist.osaka-u.ac.jp}
\date{2019年8月20日}

\begin{document}
\maketitle

\begin{equation}
  f  = \max \{U_{\rm cpu}, U_{\rm memory} \}
  \label{eq:objective_function}
\end{equation}

\begin{table}[h]
 \caption{}
 \label{}
 \centering
  \begin{tabular}{ccc}
   \hline
   Idle Timer & CPU使用率 & メモリ使用率 \\
   \hline \hline
   $\alpha$ & 60\% & 30\% \\
   $\beta$ & 50\% & 50\% \\
   $\gamma$ & 10\% & 70\% \\
   \hline
  \end{tabular}
\end{table}


% ここで、今回の評価では、CPU負荷はIdle Timerの値に対して広義単調減少(単調非増加)でありかつ、メモリ負荷はIdle Timerの値に対して広義単調増加(単調非減少)であるため、目的関数$f$を以下の式(\ref{eq:objective_function_simple})のように書き換えることも可能である。
% \begin{equation}
%   f  = U_{\rm cpu} - U_{\rm memory}
%   \label{eq:objective_function_simple}
% \end{equation}

    % \begin{figure}[htbp]
    %   \centering
    %   \includegraphics[width=0.7\hsize]{.pdf}
    %   \caption{}
    %   \label{}
    % \end{}

%
% \begin{table}[h]
%  \caption{}
%  \label{}
%  \centering
%   \begin{tabular}{ccc}
%    \hline
%    Source & Destination & The number of signaling occurrences \\
%    \hline \hline
%    Connected & Connected & $s_{\rm MME}^{\rm c \to \rm c}$ \\
%    Connected Inactive & Connected Inactive & $s_{\rm MME}^{\rm ci \to \rm ci}$ \\
%    Connected & Connected Inactive & $s_{\rm MME}^{\rm c \to \rm ci}$ \\
%    Connected Inactive & Connected & $s_{\rm MME}^{\rm ci \to \rm c}$ \\
%    Connected Inactive & Idle & $s_{\rm MME}^{\rm ci \to \rm i}$ \\
%    Idle & Connected  & $s_{\rm MME}^{\rm i \to \rm c}$ \\
%    \hline
%   \end{tabular}
% \end{table}
%
%
% \begin{equation}
%   c_h  =
%   \begin{cases}
% 		\frac{1}{T_h} \cdot s_{\rm MME}^{\rm c \to \rm c} & \text{if $T_h \le T^{\rm ci}$} \\
%     \frac{1}{T_h} \cdot (s_{\rm MME}^{\rm ci \to \rm c} + s_{\rm MME}^{\rm c \to \rm ci}) \cdot d_h  + \frac{1}{T_h} \cdot s_{\rm MME}^{\rm ci \to \rm ci} \cdot (1 - d_h) & \text{if $T^{\rm ci} < T_h \le T^{\rm i}$} \\
%     \frac{1}{T_h} \cdot (s_{\rm MME}^{\rm i \to \rm c} + s_{\rm MME}^{\rm c \to \rm ci} + s_{\rm MME}^{\rm ci \to \rm i}) & \text{otherwise}
%   \end{cases}
%   \label{eq:attach_detach}
% \end{equation}




%   \section{今後の予定}
%   \begin{itemize}
%     \item IdleTimerの設定方法を決める良いアルゴリズムがないか調査する。
%   \end{itemize}
%
% \section*{\addcontentsline{toc}{section}{参考文献}}
% \bibliographystyle{IEEEtran}
% \bibliography{/Users/t-adachi/Documents/study/Bibliography/bib/hpt_core_network/myBib/LABbiblio,/Users/t-adachi/Documents/study/Bibliography/bib/hpt_core_network/Study_Group_Bibtex/bib/hptCoreNetwork_Study}
\end{document}
